\documentclass[letterpaper, 11pt]{article} 
\usepackage{multicol}
\usepackage{amsmath}
\usepackage{hyperref}
\usepackage[margin=2cm]{geometry}

\setlength{\parskip}{0.2cm}
\setlength{\parindent}{0pt}


\title{Building a Resilient Control Scheme over Unreliable Wireless Networks}
\author{Chandler Swift \\ \\
  Spring--Fall 2019 \\
  Computer Science Department \\
  Advisor: Dr. Peter A. H. Peterson \\
  Computer Science Department \\
  University of Minnesota, Duluth
}
\date{November 30, 2019}
\begin{document}

\maketitle
\section{Introduction}
\begin{multicols}{2}
% TODO: don't announce the topic here.
The purpose of this study is to develop a method of network communication and
control of a remotely drivable research robot to enable effective network
control. This can be thought of as a specific example of the general problem
of establishing reliable communication over an unreliable network, such a
last-mile internet connection over failing cable or a satellite internet
connection, and will build off other efforts to minimize the effects of
latency such as those applied in game development (Naone, 2007) and in other
cases such as Voice over IP calling (Brunstrum,  et al., 2016).

Specifically, in our use case, we have difficulty reliably driving the robot
across campus, as it must transition between access points, a task that takes
potentally ten to fifteen seconds, in which time control and communication to
the robot are lost. The way the control system for the robot was designed at
the beginning of this research meant that the robot would continue to drive
even after connection was lost, potentially running into obstacles or
otherwise causing itself damage.

With the end goal of navigating across campus with resilience to changing
wireless access points, we will attempt to both minimize the time and the
effect of the transition time between those access points by changing the
hardware and software networking configuration of the Raspberry Pi which
acts as a network server for the robot.
\end{multicols}

\section{Methods}
\begin{multicols}{2}
Of the seven layers in the OSI networking model, there are several which we
can attempt to improve. We can refine the physical layer by adding extra or
more powerful wireless hardware, which should increase the duration we will
remain successfully connected to one access point, and minimize the frequency
of transitions. Additionally, some adapters will likely transition somewhat
more quickly than others. Having written our own control protocol on top of
the WebSocket protocol\cite{protocol}, we also explored modifications at the
protocol layer to minimize the disruption of access point handovers.

We obtained three adapters for the hardware portion of this test. First, we
use the Canakit-branded wireless adapter that came bundled with a purchased
Raspberry Pi\cite{canakit}, which is built around a MediaTek RT5370 chipset,
and supported natively by the Linux kernel. Initially this adapter had been
directly connected to the Raspberry Pi inside the frame. However, given the
detrimental effect which the metal frame is likely to have on wireless 
performance from inside, tests were run with this adapter both inside and
outside of the robot. Secondly, we purchased a BrosTrend high-gain dual-band
wireless adapter as a higher-end model to test\cite{brostrend} (built around
the RealTek 8812au chipset, which is supported natively by the Linux kernel
as of v4.19). And finally, to gain additional diversity in selection, we
purchased a third adapter: a TP-Link Archer T2UHP \cite{archer}. However,
we were unable to get this adapter to work with the Raspberry Pi, nor with
any other computer we tried, and so this was not used.

With each adapter configuration (Canakit adapter inside the frame; Canakit
adapter mounted on the top of the robot; BrosTrend adapter mounted on top
of the robot), we proceeded to make runs down the length of the third
floor of Heller Hall. Runs were conducted during weekends, where there were
fewer people present to create additional traffic to potentially disturb
results. At a consistent speed of approximately one mile per hour (so chosen
due to this being approximately the top speed of the robot), the robot was
pushed down the hall while sending ICMP \texttt{echo} packets to a computer
on the building's wired network. From this, the response time, wall clock
time, and currently connected access point were measured, 5 times per second,
until the end of the hallway was reached. Each adapter was tested over the
course of four runs using a custom-written script\cite{hallway-script}.

We also tested protocol-layer improvements. This was done by adding a
watchdog timer function\cite{watchdog} into the Raspberry Pi's web server,
where if a command was not received in 200 milliseconds (commands are
normally issued ten times per second, or every 100 milliseconds) the drive
system would be disabled until a new command was received. It is noteworthy
that the robot's hardware controller does have a watchdog timer built in, but
the documentation does not suggest that it is configurable from its default
value of ten seconds, during which time the robot can travel a dangerous
distance.

Evaluation of the effectiveness of this method was done by driving the robot
at its full speed down the hall. When it passed a marked "starting line",
the connection to the robot was cut (specifically, the websocket server which
connects the client to the robot was interrupted from the console from which
it was run), and the distance that the robot proceeded to travel was measured, 
and averaged over eight runs. This was done for both the software as it had
previously existed, and for the software with the watchdog code added.
\end{multicols}

\section{Results}
\begin{multicols}{2}
The hallway packet loss testing showed substantial results 
results results results results results results results results results results
results results results results results results results results results results
results results results results results results results results results results
results results results results results results results results results results
results results results results results results results results results results
results results results results results results results results results results
results results results results results results results results results results
results results results results results results results results results results
results results results results results results results results results results
results results results results results results results results results results
results results results results results results results results results results
results results results results results results results results results results
results results results results results results results results results results
results results results results results results results results results results
results results results results results results results results results results
results results results results results results results results results results
results results results results results results results results results results
results results results results results results results results results results
results results results results results results results results results results
results results results results results results results results results results
results results results results results results results results results results
results results results results results results results results results results
results results results results results results results results results results
\end{multicols}

\section{Discussion}
\begin{multicols}{2}
discussion discussion discussion discussion discussion discussion discussion
discussion discussion discussion discussion discussion discussion discussion
discussion discussion discussion discussion discussion discussion discussion
discussion discussion discussion discussion discussion discussion discussion
discussion discussion discussion discussion discussion discussion discussion
discussion discussion discussion discussion discussion discussion discussion
discussion discussion discussion discussion discussion discussion discussion
discussion discussion discussion discussion discussion discussion discussion
discussion discussion discussion discussion discussion discussion discussion
discussion discussion discussion discussion discussion discussion discussion
discussion discussion discussion discussion discussion discussion discussion
discussion discussion discussion discussion discussion discussion discussion
discussion discussion discussion discussion discussion discussion discussion
discussion discussion discussion discussion discussion discussion discussion
discussion discussion discussion discussion discussion discussion discussion
discussion discussion discussion discussion discussion discussion discussion
discussion discussion discussion discussion discussion discussion discussion
\end{multicols}


\section{Conclusions}
\begin{multicols}{2}
Conclusions/Further Direction
conclusions conclusions conclusions conclusions conclusions conclusions conclusions 
conclusions conclusions conclusions conclusions conclusions conclusions conclusions 
conclusions conclusions conclusions conclusions conclusions conclusions conclusions 
conclusions conclusions conclusions conclusions conclusions conclusions conclusions 
conclusions conclusions conclusions conclusions conclusions conclusions conclusions 
conclusions conclusions conclusions conclusions conclusions conclusions conclusions 
conclusions conclusions conclusions conclusions conclusions conclusions conclusions 
conclusions conclusions conclusions conclusions conclusions conclusions conclusions 
conclusions conclusions conclusions conclusions conclusions conclusions conclusions 
conclusions conclusions conclusions conclusions conclusions conclusions conclusions 
conclusions conclusions conclusions conclusions conclusions conclusions conclusions 
conclusions conclusions conclusions conclusions conclusions conclusions conclusions 
In the future, more avenues for research...kernel...etc
\end{multicols}


\bibliography{report}
\bibliographystyle{unsrt}

\end{document}
